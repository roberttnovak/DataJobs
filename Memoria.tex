% Options for packages loaded elsewhere
\PassOptionsToPackage{unicode}{hyperref}
\PassOptionsToPackage{hyphens}{url}
%
\documentclass[
]{article}
\usepackage{lmodern}
\usepackage{amssymb,amsmath}
\usepackage{ifxetex,ifluatex}
\ifnum 0\ifxetex 1\fi\ifluatex 1\fi=0 % if pdftex
  \usepackage[T1]{fontenc}
  \usepackage[utf8]{inputenc}
  \usepackage{textcomp} % provide euro and other symbols
\else % if luatex or xetex
  \usepackage{unicode-math}
  \defaultfontfeatures{Scale=MatchLowercase}
  \defaultfontfeatures[\rmfamily]{Ligatures=TeX,Scale=1}
\fi
% Use upquote if available, for straight quotes in verbatim environments
\IfFileExists{upquote.sty}{\usepackage{upquote}}{}
\IfFileExists{microtype.sty}{% use microtype if available
  \usepackage[]{microtype}
  \UseMicrotypeSet[protrusion]{basicmath} % disable protrusion for tt fonts
}{}
\makeatletter
\@ifundefined{KOMAClassName}{% if non-KOMA class
  \IfFileExists{parskip.sty}{%
    \usepackage{parskip}
  }{% else
    \setlength{\parindent}{0pt}
    \setlength{\parskip}{6pt plus 2pt minus 1pt}}
}{% if KOMA class
  \KOMAoptions{parskip=half}}
\makeatother
\usepackage{xcolor}
\IfFileExists{xurl.sty}{\usepackage{xurl}}{} % add URL line breaks if available
\IfFileExists{bookmark.sty}{\usepackage{bookmark}}{\usepackage{hyperref}}
\hypersetup{
  pdftitle={Untitled},
  pdfauthor={Geovanny Risco y Robert Novak},
  hidelinks,
  pdfcreator={LaTeX via pandoc}}
\urlstyle{same} % disable monospaced font for URLs
\usepackage[margin=1in]{geometry}
\usepackage{graphicx,grffile}
\makeatletter
\def\maxwidth{\ifdim\Gin@nat@width>\linewidth\linewidth\else\Gin@nat@width\fi}
\def\maxheight{\ifdim\Gin@nat@height>\textheight\textheight\else\Gin@nat@height\fi}
\makeatother
% Scale images if necessary, so that they will not overflow the page
% margins by default, and it is still possible to overwrite the defaults
% using explicit options in \includegraphics[width, height, ...]{}
\setkeys{Gin}{width=\maxwidth,height=\maxheight,keepaspectratio}
% Set default figure placement to htbp
\makeatletter
\def\fps@figure{htbp}
\makeatother
\setlength{\emergencystretch}{3em} % prevent overfull lines
\providecommand{\tightlist}{%
  \setlength{\itemsep}{0pt}\setlength{\parskip}{0pt}}
\setcounter{secnumdepth}{-\maxdimen} % remove section numbering

\title{Untitled}
\author{Geovanny Risco y Robert Novak}
\date{3/1/2022}

\begin{document}
\maketitle

\begin{verbatim}
##  [1] "index"             "Job Title"         "Salary Estimate"  
##  [4] "Job Description"   "Rating"            "Company Name"     
##  [7] "Location"          "Headquarters"      "Size"             
## [10] "Founded"           "Type of ownership" "Industry"         
## [13] "Sector"            "Revenue"           "Competitors"
\end{verbatim}

\hypertarget{descripciuxf3n-del-dataset}{%
\section{Descripción del Dataset}\label{descripciuxf3n-del-dataset}}

El dataset que hemos escogido está recogido mediante webscraping en
distintas plataformas sobre ofertas de empleo relacionadas con los datos
en Estados Unidos. Hemos escogido este dataset principalmente por dos
razones.

\begin{enumerate}
\def\labelenumi{\arabic{enumi}.}
\item
  Es un dataset bastante desordenado en el que da lugar a hacer procesos
  de limpieza de distinto tipo ideal para asentar los conceptos tratados
  en la asignatura
\item
  Nos parece interesante conocer el mercado laboral de las distintas
  profesiones a las que podríamos aspirar tras la finalización del
  máster y la demanda, aunque sea en un país extranjero.
\end{enumerate}

Nuestro principal objetivo con el dataset es contestar a distintas
preguntas relacionadas con el salario y distintas variables que se
proporcionan en el dataset:

Aquí concretar un poco algunas de ellas

\hypertarget{integraciuxf3n-y-selecciuxf3n-de-los-datos-de-interuxe9s-a-analizar}{%
\section{Integración y selección de los datos de interés a
analizar}\label{integraciuxf3n-y-selecciuxf3n-de-los-datos-de-interuxe9s-a-analizar}}

\hypertarget{limpieza-de-datos}{%
\section{Limpieza de datos}\label{limpieza-de-datos}}

\hypertarget{tratamiento-de-las-distintas-variables-del-dataset}{%
\subsection{Tratamiento de las distintas variables del
dataset}\label{tratamiento-de-las-distintas-variables-del-dataset}}

En primer lugar, hemos realizado un análisis del dominio de las
variables a partir de las cuáles hemos hecho hecho las siguientes
observaciones:

\begin{enumerate}
\def\labelenumi{\arabic{enumi}.}
\item
  Se utiliza el \texttt{-1} para indicar valores faltantes.
  Adicionalmente, existen columnas que tienen un valor faltante que se
  representa de forma distinta a \texttt{-1} por la forma en la que se
  han extraído los datos. En la limpieza hemos tratado todos esos casos
  y representado los valores faltantes de forma homogénea mediante
  \texttt{NA}, que es la forma de representar los valores faltantes en R
  y gracias al cuál podemos hacer operaciones para algunas de las
  funciones donde se tienen en cuenta los valores faltantes.
\item
  La columna \texttt{Job\ title} tiene una gran diversidad de trabajos
  con una mínima variación en la que es interesante tratarlos como un
  mismo trabajo. Para ello, hemos definido un subconjunto de trabajos a
  partir del cuál tratar como iguales las variantes. Ese subconjunto son
  los que consideramos principales : \{ data scientist, data engineer,
  data analyst, machine learning\}. Así, por ejemplo, un trabajo de
  e-commerce data analyst o uno de RFP data analyst será tratado bajo la
  categoría de data analyst. Aquellos trabajos muy específicos en los
  cuáles no se engloba bajo ninguna de las categorías anteriores los
  consideramos muy específicos y, al no ser un número muy elevado hemos
  decidido eliminarlos del dataset.
\item
  La variable \texttt{Company\ name} tiene la información del rating.
  Hemos eliminado esa redundancia
\item
  Hemos añadido una nueva variable binaria a partir de \texttt{Location}
  y \texttt{Headquarters} para ver aquellas ofertas de trabajo en la que
  la cede central de la empresa está en el mismo sitio que la oferta
\item
  Algunas variables como \texttt{Salary\ Estimate}, \texttt{Size} y
  \texttt{Revenue} contienen información que pueden ser aprovechadas
  mejor separándolas en más columnas a partir de las cuáles sacar más
  información. Así, las hemos separado en más columnas. Una para los
  rangos mínimos, otro para los rangos máximos y otra para los medios.
\item
  \texttt{Salary\ Estimate} puede ser considerada una variable
  cuantitativa ya que, aunque se proporcione un rango variable para
  todas las ofertas, la realidad es que el salario no es un rango sino
  un valor concreto dado por un dominio continuo. La decisión que hemos
  tomado para solucionar esto es considerar el punto medio del rango
  proporcionado como el salario de la oferta. Esta solución es una
  aproximación ya que dos ofertas con mismos rangos tendrían el mismo
  salario y no tendría por qué ser considerados como el mismo. O,
  incluso, dos salarios con rangos distintos pero con una cierta
  intersección podrían tener en la realidad el mismo salario pero no tal
  como lo hemos tratado. Sin embargo, aunque lo ideal sería hacer un
  estudio externo sobre la distribución del salario dado el rango, la
  empresa particular, etc. Al no disponer de esa información asumimos
  esta simplificación.
\item
  \texttt{Size} y \texttt{Revenue} deben ser consideradas para análisis
  posteriores como variables ordinales ya que su dominio corresponde a
  categorías no solapadas en el que el orden importa.
\item
  La variable \texttt{Job\ Description} es una variable muy interesante
  a partir de la cuál se puede obtener información interesante como, por
  ejemplo, los lenguajes de programación exigidos por la oferta, los
  años de experiencia necesario, etc. Sin embargo, esta información está
  presente de forma muy variable de observación en observación lo que
  hace difícil su extracción para los conocimientos que tenemos
  actualmente (aún no hemos tenido ninguna asignatura de NLP). Por ello,
  hemos decidido no tratarla más que para comprobar casos atípicos por
  si hay alguna información interesante que pudiera explicarlos
\end{enumerate}

\hypertarget{tratamiento-de-los-valores-extremos-de-la-variable-de-interuxe9s-salary-estimate-med}{%
\subsection{\texorpdfstring{Tratamiento de los valores extremos de la
variable de interés
\texttt{Salary\ Estimate\ Med}}{Tratamiento de los valores extremos de la variable de interés Salary Estimate Med}}\label{tratamiento-de-los-valores-extremos-de-la-variable-de-interuxe9s-salary-estimate-med}}

\includegraphics{Memoria_files/figure-latex/unnamed-chunk-2-1.pdf}

En el análisis ue hemos realizado, hemos observado que la variable
\texttt{Salary\ Estimate\ Med} tiene una distribución de valores
bastante consistente, lo cual podemos comprobar dado el pequeño tamaño
de su rango intercuartílico (caja). No obstante, vemos como existen
puntos fuera de este rango y de los \emph{whiskers}, los conocidos como
\emph{outliers}. En concreto, son dos puntos: 43.5 y 271.5.

Explorando los casos en el que el salario es atípicamente bajo hemos
encontrado que se corresponde con posiciones en las que no se piden
experiencia (lo podemos comprobar por la variable
\texttt{Job\ Description}) y, además, se encuentran en ciudades de
Estados Unidos poco punteras (tecnologicamente hablando), como por
ejemplo ``Lincoln'', ``Arlington'', ``Saint Paul'', etc. Por tanto,
podemos concluir que estas variables no se tratan de errores en el
dataset, sino de valores totalmente legítimos que debemos tener en
cuenta para las pruebas estadísticas posteriores.

En el caso del valor atípico alto podemos observar lo contrario, estas
observaciones están asociadas a:

\begin{itemize}
\tightlist
\item
  Posiciones para managers, seniors o PhDs. Es decir, se requiere
  formación y experiencia para optar a este sueldo
\item
  Ciudad muy punteras en el ambito tecnológico. Ex. ``Seattle'', ``New
  York'', ``Washington'', etc.
\item
  Empresas muy top. Ex. ``Roche'', ``Aztrazeneca'', ``Maxar
  Technologies'', etc.
\end{itemize}

Dado que estos salarios tan altos son totalmente justificables (por las
razones planteadas, entre otras), hemos mantenido los registros para las
pruebas estadísticas posteriores.

\hypertarget{anuxe1lisis-de-datos}{%
\section{Análisis de datos}\label{anuxe1lisis-de-datos}}

\hypertarget{comprobaciuxf3n-de-normalidad-de-la-variable-salary-estimate-med}{%
\subsection{\texorpdfstring{Comprobación de normalidad de la variable
\texttt{Salary\ Estimate\ Med}}{Comprobación de normalidad de la variable Salary Estimate Med}}\label{comprobaciuxf3n-de-normalidad-de-la-variable-salary-estimate-med}}

La variable \texttt{Salary\ Estimate\ Med} representa la variable
cuantitativa de interés según lo explicado en el apartado de limpieza de
datos.

A continuación proporcionamos dos gráficas para comprobar visualmente,
previo al test de normalidad, si la variable tiene apariencia de ser
normal.

En vista de los gráficos se puede apreciar que no parece que la
distribución del salario sea normal. Comprobemoslo de manera formal
usando el test de shapiro

El p valor es muy bajo por lo que se puede rechazar la hipótesis nula de
que la variable sigue una distribución normal. Hay que tener en cuenta
este hecho para los análisis relacionados que presu

\hypertarget{anuxe1lisis-de-la-normalidad-de-la-variable-salary}{%
\subsection{Análisis de la normalidad de la variable
salary}\label{anuxe1lisis-de-la-normalidad-de-la-variable-salary}}

\hypertarget{representaciuxf3n-de-los-resultados-a-partir-de-tablas-y-gruxe1ficas}{%
\section{Representación de los resultados a partir de tablas y
gráficas}\label{representaciuxf3n-de-los-resultados-a-partir-de-tablas-y-gruxe1ficas}}

\hypertarget{cuxf3digo}{%
\section{Código}\label{cuxf3digo}}

\end{document}
